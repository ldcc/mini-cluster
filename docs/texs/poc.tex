%%
%% Author: ldc
%% 6/17/19
%%

%% Preamble
\documentclass[usenames,dvipsnames,tikz]{article}

%% Dependences
\usepackage{amsmath}
\usepackage{tikz}
\usepackage{xeCJK}
\usepackage[T1]{fontenc}
\usepackage[utf8]{inputenc}
\usepackage{color}
\usepackage{xcolor}
\usepackage[b4paper,landscape,nohead,margin=2cm]{geometry}
\usetikzlibrary{matrix,shapes,mindmap,shadows,backgrounds}

%% Definition
\definecolor{shallowyellow}{RGB}{245,245,204}
\def\anno#1#2{\node [anno] at (#2) {\textbf{\large #1}}}

%% Settings
\pagestyle{empty}
\pagecolor{shallowyellow}
\tikzstyle{root}+=[font=\Huge]
\tikzstyle{node}+=[concept color=teal]
\tikzstyle{cons}+=[concept color=purple]
\tikzstyle{chain}+=[concept color=darkgray]
\tikzstyle{anno}=[annotation,fill=black!50,above,xshift=-5cm,text width=8cm,inner sep=4mm]
\tikzstyle{group}+=[mindmap, grow cyclic,concept color=darkgray,every node/.append style={concept,circular drop shadow}]
\tikzstyle{catethe}=[matrix of math nodes,row sep=3em,column sep=4em,minimum width=2em,text=black]
\tikzstyle{every node}+=[text=white,scale=1.2]
\tikzstyle{level 1 concept}+=[level distance=5.2cm,font=\Large]
\tikzstyle{level 2 concept}+=[level distance=3.8cm,font=\large]
%\tikzstyle{level 3 concept}+=[level distance=2.3cm,font=\scriptsize,rotate=30]
%\tikzstyle{level 4 concept}+=[level distance=1.6cm,font=\tiny]

%% Document
\begin{document}
  \begin{tikzpicture}
    \path[group]
    node[root] (chain1) at (0,0) {$Chain1$}
    child[node] {node (node1) {$Node1$}}
    child[node] {node (node2) {$Node2$}}
    child[cons] {node (pow) {$PoW$}};
    \path[group]
    node[root] (chain2) at (10,-7) {$Chain2$}
    child[node] {node (node3) {$Node3$}}
    child[node] {node (node4) {$Node4$}}
    child[node] {node (node5) {$Node5$}}
    child[cons] {node (pos) {$PoS$}};
    \path[group]
    node[root] (chain3) at (-12,-4) {$Chain3$}
    child[cons,clockwise from=90] {node (poa) {$PoA$}};
    \path (node1) to [circle connection bar switch color = from (teal) to (darkgray)] (chain2);
    \path (node1) to [circle connection bar switch color = from (teal) to (darkgray)] (chain3);
    \path (node3) to [circle connection bar switch color = from (teal) to (darkgray)] (chain3);
    \path (chain1) to [circle connection bar switch color = from (darkgray) to (darkgray)] (chain2);
    \path (chain1) to [circle connection bar switch color = from (darkgray) to (darkgray)] (chain3);
    \anno{假设一个运行于 BCID2.0 下的拓扑网络中的所有节点呈现以下分布}{chain1.north};
  \end{tikzpicture}
  \newpage

  \begin{tikzpicture}
    \path[group,node,level 1/.append style={sibling angle=120}]
    node[root] (node) {$Node1$}
    child[chain] {node {$Chain1$} child[cons] {node {$PoW$}}}
    child[chain] {node {$Chain2$} child[cons] {node {$PoS$}}}
    child[chain] {node {$Chain3$} child[cons] {node {$PoA$}}};
    \anno{那么对于单个节点而言,比如 Node1 ,它的运行环境是这样的}{node.west};
  \end{tikzpicture}
  \newpage

  \begin{tikzpicture}
    \path[group]
    node[root] (group) {ChainX}
    child[node] {node {$Node1$} child[node] {node {$Node2$}}}
    child[cons] {node {$PoX$}};
    \anno{对于一个局部共识 G1,当节点 Node2 尝试通过连接 Node1 加入到共识中时}{group.north};
  \end{tikzpicture}

%  \begin{tikzpicture}
%    \matrix(m)[catethe] {
%    Node_1 & Node_2 \\
%    Chain_X & x       \\};
%    %    \path[-stealth]
%    %    (m-1-1) edge node [left]
%  \end{tikzpicture}


\end{document}
